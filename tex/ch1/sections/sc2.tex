

\section{Halbleitertypen}
	\subsection{Halbleiter und Metalle}
	\begin{itemize}
		\item Metalle sind grundsätzlich Kaltleiter $\rightarrow$ $T=0K \; \Rightarrow \; \sigma_M \rightarrow \infty$
		\begin{itemize}
			\item T $\uparrow \Rightarrow \sigma_M \downarrow, \; \sigma_M >>0$
		\end{itemize}
		\item Halbleiter sind grundsätzlich Heissleiter
		\begin{itemize}
			\item $T\uparrow \Rightarrow \sigma_{HL} \uparrow, \text{ aber } \sigma_{HL} << \sigma_M$
		\end{itemize}
	\end{itemize}
	
	\subsection{Elementhalbleiter} \label{ss_hltyp_ehl}
	Elementhalbleiter sind Halbleiter die nur aus einem Element bestehen.
	\subsection{Verbindungshalbleiter} \label{ss_hltyp_vhl}
	Verbindungshalbleiter sind immer 50/50 aufgeteilt, d.h. am Beispiel eines GaN (Galliumnitrid) III-V Verbindungshalbleiter ist das Kristallgitter so aufgebaut dass au jedes Galliumatom ein Stickstoffatom und dann wieder ein Galliumatom folgt. 
	\subsection{Legierungshalbleiter} \label{ss_hltyp_lhl}
	Legierungshalbleiter sind im Gegensatz zu Verbindungshalbleiter beliebig mischbar. Grundsätzlich sind alle Verhältnisse denkbar die sich ineinander lösen lassen.So können auch Kristallgitter entstehen, in denen gleiche Atome direkt benachbart sind.
	


