\section{Geschichtliche Facts}
\begin{itemize}
	\item Kommerzieller Start: 1948 (Bell-Labs)
	\begin{itemize}
		\item Zunächst mit Germanium aufgrund besserer Eigenschaften
		\item Erstamls BPT (Bipolartransistor) genutzt für Verstärker- und HF-Technik. Der BPT sticht hier mit einer höheren Löcher- und Elektronenbeweglichkeit hervor. Insbesondere für eine hohe Löcherbeweglichkeit ist Germanium das beste bekannte Material.
	\end{itemize}
	\item Si dominiert den Markt seit 1965
	\item Übergang von $>100nm$ zu $<100nm$ ungefähr im Jahr $\sim 2000$
	\item Der erste Transistor wurde 1938 von Robert Wichard Pohl und Rudolf Hilsch aus Kalium-Bromid hergestellt. Das Material war aber ungeeignet da wasserlöslich.
	\item Erstes kommerzielles transistorbasiertes Produkt: Texas Instruments Taschenrechner
\end{itemize}

\section{Abkürzungen/Bezeichnungen}
\begin{itemize}
	\item CVD : \underline{C }hemical \underline{V}apor \underline{D}eposition
	\item sputter : zerstäuben (engl.)
	\item epi : Epitaxie  = gerichtetes Aufwachsen (gr.)
	\item Flache Störstelle: Störstelle nahe der Leitungs/Valenzbandkante
	\item Tiefe Störtstelle: Störstelle die tief in der Bandlücke liegt
	\item BJT: \underline{B}ipolar \underline{J}unction \underline{T}ransistor
	\item Transistor: \underline{Trans}fer Res\underline{istor}
	\item SRAM: \underline{S}tatic \underline{R}andom \underline{A}ccess \underline{M}emory
	\item DRAM: \underline{D}ynamic \underline{R}andom \underline{A}ccess \underline{M}emory
	\item Iso: gleich
	\item Trop: Raum
	\item Isotrop leitend: Im Raum in alle Richtungen gleich gut leitend
	\item Anisotrop leitend: Im Raum in unterschiedlichen Richtungen unterschiedlich gut leitend
	\item Morphologie: Struktur (Form)
	\item Homo: Gleich im Sinne von ähnlich
	\item Iso: Gleich im Sinne von identisch
	\item Homomorphismus: Ähnlich in der Struktur
	\item Isomorphismus: Identisch in der Struktur
	\item pe: plasma enhanced
	\item Inertgas: Gase die sehr reaktionsträge sind z.B. Edelgase oder Stickstoff
\end{itemize}

\section{Nice to know}
\begin{itemize}
	\item Der BPT ist sehr viel schneller als der MOSFET und wird daher häufig noch in der HF Technik eingesetzt.
	\item MOSFETs können als normally on oder normally off gebaut werden. Am gebräuchlichsten ist normally off.
	\item Ein Areniusplot stellt die Temperatur und die Leitfähigkeit ins Verhältnis.
	\item Eine Elektronenpaarbindung ist eine kovalente Bindung
	PNP Transistoren sind üblicherweise breiter als NPN Transistoren, da die Beweglichkeit von Löchern kleiner als die von Elektronen ist $\rightarrow$ bei gleicher Baugröße treibt ein PNP Transistor weniger Strom als ein NPN Transistor
	\item Im Periodensystem abwärts werden die Atome immer größer
	\item Zinn hat bis $13^\circ C$ die gleiche Kristallstruktur wie Silizium, danach baut es sich zu sogenanntem Beta-Zinn um, das dieselbe Struktur wie Metall hat.
	\item Bei Isotopen sind die Anzahl der Protonen gleich der Anzahl der Neutronen gleich der Anzahl der Elektronen. Sie sind damit die Grundzustände von Atomen
	\item Größenordnung zwischen Atomen: Angström
	\item Größenordnung im Atomkern: $10^{-15}m$
	\item Systemarten
			\begin{table}[H]
				\centering
				\begin{tabular}{c c}
					Systemart & Anzahl Teilchen \\ \hline
					Mikroskopisch & Maximal 3 Teilchen \\
					Mesoskopisch & $\sim$ 100 - 1000 Teilchen\\
					Makroskopisch & $\sim$ Millionen und mehr Teilchen
				\end{tabular}
			\end{table}
			\vspace{-0.5cm}
			Bei Makroskopischen Systemen wird statistisch gearbeitet. D.h. es wird der Mittelwert des Systems untersucht um so das Problem auf ein Ein-Teilchen Problem zu reduzieren.
		\item Die Intensität einer Welle hängt vom Quadrat der Amplitude ab
		\item Die Energie einer EM-Welle hängt von der Frequenz und Intensität ab
		\item Grenzfrequenz lässt sich mit klassischer Physik nicht richtig erklären
		\item Einstein hat bei der Untersuchung des photoelektrioschen Effekts den Begriff des Photons geprägt
		\item Mehr Intensität $\Rightarrow$ mehr Photonen $\Rightarrow$ Photon $\cong h \cdot f$ $\Rightarrow$ abhängig von der Frequenz
		\item Pauliprinzip: Jeder Zustand (auch einzelne Spins) ist entweder frei oder durch nur ein Elektron besetzbar
		\item Elemente gehen nur solche chemische Verbindungen ein, dass sie möglichst in Edelgaskonfiguration kommen, also abgeschlossene Außenschalen haben
			  \item Im Periodensystem stehen Metalle meistens in sogenannten Übergangsgruppen bzw. Nebengruppen. Hier ist es schwierig zu sagen ob ein Element Elektronen aufnehmen oder abnehmen sollte um Edelgaskonfiguration zu erreichen
			  \item In Materie befinden sich ca. $10^{23} cm^{-3}$ Atome, gibt nun jedes dieser Atome 1-2 Elektronen in den Fermi-See abgibt erhalten wir ca. $1-2 \cdot 10^{23} cm^{-3}$ Elektronen $\Rightarrow$ sehr hohe Leitfähigkeit
			  \item Zu 99\% besteht Materie aus nichts
			  \item Komplette Bandstruktur wäre ein 7D Bild
			  \item Metalle werden nach Dichte klassifiziert (Leicht- und Schwermetall)
			  \item In einem Metall ist der letzte bei $T = 0K$ noch besetzte energetische Zustand das Fermi-Niveau
			  \item Die Elektronen die bei einem Metall hauptsächlich für die Leitfähigkeit eine Rolle spielen sind die Nahe des Fermi-Niveaus, da dort die Hauptmasse der Elektronen ist
			  \item Natrium und Chlor diffundiert gerne in Glas
			  \item Struktur der Größe $x\;nm$ muss mit Licht der Wellenlänge $\lambda \approx x \; nm$ bestrahlt werden zur mikroskopischen Untersuchung/Lithographie
			  \item Mensch sieh grün und gelb am besten $\sim 600nm$
		\item $1cm^3$ Si besteht aus $\sim 10^{23}$ Atomen
		\item Minimale Reinheit die Si für die Produktion von Halbleitern vorweisen muss: $0.999999999$, wobei die Genauigkeit auf mind. 8 Nachkommastellen gegeben sein muss.
	\end{itemize}

\section{Literaturempfehlungen}
\begin{itemize}
	\item Sze - Physics of Semiconductor Devices \& Semiconductor Devices – Physics and Technology, John Wiley \\
	\item Hoffmann - Systemintegration, Oldenburg
\end{itemize}

\section{Periodensystem}
\newcommand{\CommonElementTextFormat}[4]
{
  \begin{minipage}{2.2cm}
    \centering
      {\textbf{#1} \hfill #2}%
      \linebreak \linebreak
      {\textbf{#3}}%
      \linebreak \linebreak
      {{#4}}
  \end{minipage}
}

\newcommand{\NaturalElementTextFormat}[4]
{
  \CommonElementTextFormat{#1}{#2}{\LARGE {#3}}{#4}
}

\newcommand{\OutlineText}[1]
{
\ifpdf
  % Couldn't find a nicer way of doing an outline font with TikZ
  % other than using pdfliteral 1 Tr
  %
  \pdfliteral direct {0.5 w 1 Tr}{#1}%
  \pdfliteral direct {1 w 0 Tr}%
\else
  % pstricks can do this with \pscharpath from pstricks
  %
  \pscharpath[shadow=false,
    fillstyle=solid,
    fillcolor=white,
    linestyle=solid,
    linecolor=black,
    linewidth=.2pt]{#1} 
\fi
}

\newcommand{\SyntheticElementTextFormat}[4]
{
\ifpdf
  \CommonElementTextFormat{#1}{#2}{\OutlineText{\LARGE #3}}{#4}
\else
  % pstricks approach results in slightly larger box
  % that doesn't break, so fudge here
  \CommonElementTextFormat{#1}{#2}{\OutlineText{\Large #3}}{#4}
\fi
}


\begin{tikzpicture}[font=\sffamily, scale=0.32, transform shape]

%% Fill Color Styles
  \tikzstyle{ElementFill} = [fill=yellow!15]
  \tikzstyle{AlkaliMetalFill} = [fill=blue!55]
  \tikzstyle{AlkalineEarthMetalFill} = [fill=blue!40]
  \tikzstyle{MetalFill} = [fill=blue!25]
  \tikzstyle{MetalloidFill} = [fill=orange!25]
  \tikzstyle{NonmetalFill} = [fill=green!25]
  \tikzstyle{HalogenFill} = [fill=green!40]
  \tikzstyle{NobleGasFill} = [fill=green!55]
  \tikzstyle{LanthanideActinideFill} = [fill=purple!25]

%% Element Styles
  \tikzstyle{Element} = [draw=black, ElementFill,
    minimum width=2.75cm, minimum height=2.75cm, node distance=2.75cm]
  \tikzstyle{AlkaliMetal} = [Element, AlkaliMetalFill]
  \tikzstyle{AlkalineEarthMetal} = [Element, AlkalineEarthMetalFill]
  \tikzstyle{Metal} = [Element, MetalFill]
  \tikzstyle{Metalloid} = [Element, MetalloidFill]
  \tikzstyle{Nonmetal} = [Element, NonmetalFill]
  \tikzstyle{Halogen} = [Element, HalogenFill]
  \tikzstyle{NobleGas} = [Element, NobleGasFill]
  \tikzstyle{LanthanideActinide} = [Element, LanthanideActinideFill]
  \tikzstyle{PeriodLabel} = [font={\sffamily\LARGE}, node distance=2.0cm]
  \tikzstyle{GroupLabel} = [font={\sffamily\LARGE}, minimum width=2.75cm, node distance=2.0cm]
  \tikzstyle{TitleLabel} = [font={\sffamily\Huge\bfseries}]

%% Group 1 - IA
  \node[name=H, Element] {\NaturalElementTextFormat{1}{1.0079}{H}{Hydrogen}};
  \node[name=Li, below of=H, AlkaliMetal] {\NaturalElementTextFormat{3}{6.941}{Li}{Lithium}};
  \node[name=Na, below of=Li, AlkaliMetal] {\NaturalElementTextFormat{11}{22.990}{Na}{Sodium}};
  \node[name=K, below of=Na, AlkaliMetal] {\NaturalElementTextFormat{19}{39.098}{K}{Potassium}};
  \node[name=Rb, below of=K, AlkaliMetal] {\NaturalElementTextFormat{37}{85.468}{Rb}{Rubidium}};
  \node[name=Cs, below of=Rb, AlkaliMetal] {\NaturalElementTextFormat{55}{132.91}{Cs}{Caesium}};
  \node[name=Fr, below of=Cs, AlkaliMetal] {\NaturalElementTextFormat{87}{223}{Fr}{Francium}};

%% Group 2 - IIA
  \node[name=Be, right of=Li, AlkalineEarthMetal] {\NaturalElementTextFormat{4}{9.0122}{Be}{Beryllium}};
  \node[name=Mg, below of=Be, AlkalineEarthMetal] {\NaturalElementTextFormat{12}{24.305}{Mg}{Magnesium}};
  \node[name=Ca, below of=Mg, AlkalineEarthMetal] {\NaturalElementTextFormat{20}{40.078}{Ca}{Calcium}};
  \node[name=Sr, below of=Ca, AlkalineEarthMetal] {\NaturalElementTextFormat{38}{87.62}{Sr}{Strontium}};
  \node[name=Ba, below of=Sr, AlkalineEarthMetal] {\NaturalElementTextFormat{56}{137.33}{Ba}{Barium}};
  \node[name=Ra, below of=Ba, AlkalineEarthMetal] {\NaturalElementTextFormat{88}{226}{Ra}{Radium}};

%% Group 3 - IIIB
  \node[name=Sc, right of=Ca, Metal] {\NaturalElementTextFormat{21}{44.956}{Sc}{Scandium}};
  \node[name=Y, below of=Sc, Metal] {\NaturalElementTextFormat{39}{88.906}{Y}{Yttrium}};
  \node[name=LaLu, below of=Y, LanthanideActinide] {\NaturalElementTextFormat{57-71}{}{La-Lu}{Lanthanide}};
  \node[name=AcLr, below of=LaLu, LanthanideActinide] {\NaturalElementTextFormat{89-103}{}{Ac-Lr}{Actinide}};

%% Group 4 - IVB
  \node[name=Ti, right of=Sc, Metal] {\NaturalElementTextFormat{22}{47.867}{Ti}{Titanium}};
  \node[name=Zr, below of=Ti, Metal] {\NaturalElementTextFormat{40}{91.224}{Zr}{Zirconium}};
  \node[name=Hf, below of=Zr, Metal] {\NaturalElementTextFormat{72}{178.49}{Hf}{Halfnium}};
  \node[name=Rf, below of=Hf, Metal] {\SyntheticElementTextFormat{104}{261}{Rf}{Rutherfordium}};

%% Group 5 - VB
  \node[name=V, right of=Ti, Metal] {\NaturalElementTextFormat{23}{50.942}{V}{Vanadium}};
  \node[name=Nb, below of=V, Metal] {\NaturalElementTextFormat{41}{92.906}{Nb}{Niobium}};
  \node[name=Ta, below of=Nb, Metal] {\NaturalElementTextFormat{73}{180.95}{Ta}{Tantalum}};
  \node[name=Db, below of=Ta, Metal] {\SyntheticElementTextFormat{105}{262}{Db}{Dubnium}};

%% Group 6 - VIB
  \node[name=Cr, right of=V, Metal] {\NaturalElementTextFormat{24}{51.996}{Cr}{Chromium}};
  \node[name=Mo, below of=Cr, Metal] {\NaturalElementTextFormat{42}{95.94}{Mo}{Molybdenum}};
  \node[name=W, below of=Mo, Metal] {\NaturalElementTextFormat{74}{183.84}{W}{Tungsten}};
  \node[name=Sg, below of=W, Metal] {\SyntheticElementTextFormat{106}{266}{Sg}{Seaborgium}};

%% Group 7 - VIIB
  \node[name=Mn, right of=Cr, Metal] {\NaturalElementTextFormat{25}{54.938}{Mn}{Manganese}};
  \node[name=Tc, below of=Mn, Metal] {\NaturalElementTextFormat{43}{96}{Tc}{Technetium}};
  \node[name=Re, below of=Tc, Metal] {\NaturalElementTextFormat{75}{186.21}{Re}{Rhenium}};
  \node[name=Bh, below of=Re, Metal] {\SyntheticElementTextFormat{107}{264}{Bh}{Bohrium}};

%% Group 8 - VIIIB
  \node[name=Fe, right of=Mn, Metal] {\NaturalElementTextFormat{26}{55.845}{Fe}{Iron}};
  \node[name=Ru, below of=Fe, Metal] {\NaturalElementTextFormat{44}{101.07}{Ru}{Ruthenium}};
  \node[name=Os, below of=Ru, Metal] {\NaturalElementTextFormat{76}{190.23}{Os}{Osmium}};
  \node[name=Hs, below of=Os, Metal] {\SyntheticElementTextFormat{108}{277}{Hs}{Hassium}};

%% Group 9 - VIIIB
  \node[name=Co, right of=Fe, Metal] {\NaturalElementTextFormat{27}{58.933}{Co}{Cobalt}};
  \node[name=Rh, below of=Co, Metal] {\NaturalElementTextFormat{45}{102.91}{Rh}{Rhodium}};
  \node[name=Ir, below of=Rh, Metal] {\NaturalElementTextFormat{77}{192.22}{Ir}{Iridium}};
  \node[name=Mt, below of=Ir, Metal] {\SyntheticElementTextFormat{109}{268}{Mt}{Meitnerium}};

%% Group 10 - VIIIB
  \node[name=Ni, right of=Co, Metal] {\NaturalElementTextFormat{28}{58.693}{Ni}{Nickel}};
  \node[name=Pd, below of=Ni, Metal] {\NaturalElementTextFormat{46}{106.42}{Pd}{Palladium}};
  \node[name=Pt, below of=Pd, Metal] {\NaturalElementTextFormat{78}{195.08}{Pt}{Platinum}};
  \node[name=Ds, below of=Pt, Metal] {\SyntheticElementTextFormat{110}{281}{Ds}{Darmstadtium}};

%% Group 11 - IB
  \node[name=Cu, right of=Ni, Metal] {\NaturalElementTextFormat{29}{63.546}{Cu}{Copper}};
  \node[name=Ag, below of=Cu, Metal] {\NaturalElementTextFormat{47}{107.87}{Ag}{Silver}};
  \node[name=Au, below of=Ag, Metal] {\NaturalElementTextFormat{79}{196.97}{Au}{Gold}};
  \node[name=Rg, below of=Au, Metal] {\SyntheticElementTextFormat{111}{280}{Rg}{Roentgenium}};

%% Group 12 - IIB
  \node[name=Zn, right of=Cu, Metal] {\NaturalElementTextFormat{30}{65.39}{Zn}{Zinc}};
  \node[name=Cd, below of=Zn, Metal] {\NaturalElementTextFormat{48}{112.41}{Cd}{Cadmium}};
  \node[name=Hg, below of=Cd, Metal] {\NaturalElementTextFormat{80}{200.59}{Hg}{Mercury}};
  \node[name=Uub, below of=Hg, Metal] {\SyntheticElementTextFormat{112}{285}{Uub}{Ununbium}};

%% Group 13 - IIIA
  \node[name=Ga, right of=Zn, Metal] {\NaturalElementTextFormat{31}{69.723}{Ga}{Gallium}};
  \node[name=Al, above of=Ga, Metal] {\NaturalElementTextFormat{13}{26.982}{Al}{Aluminium}};
  \node[name=B, above of=Al, Metalloid] {\NaturalElementTextFormat{5}{10.811}{B}{Boron}};
  \node[name=In, below of=Ga, Metal] {\NaturalElementTextFormat{49}{114.82}{In}{Indium}};
  \node[name=Tl, below of=In, Metal] {\NaturalElementTextFormat{81}{204.38}{Tl}{Thallium}};
  \node[name=Uut, below of=Tl, Metal] {\SyntheticElementTextFormat{113}{284}{Uut}{Ununtrium}};

%% Group 14 - IVA
  \node[name=C, right of=B, Nonmetal] {\NaturalElementTextFormat{6}{12.011}{C}{Carbon}};
  \node[name=Si, below of=C, Metalloid] {\NaturalElementTextFormat{14}{28.086}{Si}{Silicon}};
  \node[name=Ge, below of=Si, Metalloid] {\NaturalElementTextFormat{32}{72.64}{Ge}{Germanium}};
  \node[name=Sn, below of=Ge, Metal] {\NaturalElementTextFormat{50}{118.71}{Sn}{Tin}};
  \node[name=Pb, below of=Sn, Metal] {\NaturalElementTextFormat{82}{207.2}{Pb}{Lead}};
  \node[name=Uuq, below of=Pb, Metal] {\SyntheticElementTextFormat{114}{289}{Uuq}{Ununquadium}};

%% Group 15 - VA
  \node[name=N, right of=C, Nonmetal] {\NaturalElementTextFormat{7}{14.007}{N}{Nitrogen}};
  \node[name=P, below of=N, Nonmetal] {\NaturalElementTextFormat{15}{30.974}{P}{Phosphorus}};
  \node[name=As, below of=P, Metalloid] {\NaturalElementTextFormat{33}{74.922}{As}{Arsenic}};
  \node[name=Sb, below of=As, Metalloid] {\NaturalElementTextFormat{51}{121.76}{Sb}{Antimony}};
  \node[name=Bi, below of=Sb, Metal] {\NaturalElementTextFormat{83}{208.98}{Bi}{Bismuth}};
  \node[name=Uup, below of=Bi, Metal] {\SyntheticElementTextFormat{115}{288}{Uup}{Ununpentium}};

%% Group 16 - VIA
  \node[name=O, right of=N, Nonmetal] {\NaturalElementTextFormat{8}{15.999}{O}{Oxygen}};
  \node[name=S, below of=O, Nonmetal] {\NaturalElementTextFormat{16}{32.065}{S}{Sulphur}};
  \node[name=Se, below of=S, Nonmetal] {\NaturalElementTextFormat{34}{78.96}{Se}{Selenium}};
  \node[name=Te, below of=Se, Metalloid] {\NaturalElementTextFormat{52}{127.6}{Te}{Tellurium}};
  \node[name=Po, below of=Te, Metalloid] {\NaturalElementTextFormat{84}{209}{Po}{Polonium}};
  \node[name=Uuh, below of=Po, Metal] {\SyntheticElementTextFormat{116}{293}{Uuh}{Ununhexium}};

%% Group 17 - VIIA
  \node[name=F, right of=O, Halogen] {\NaturalElementTextFormat{9}{18.998}{F}{Flourine}};
  \node[name=Cl, below of=F, Halogen] {\NaturalElementTextFormat{17}{35.453}{Cl}{Chlorine}};
  \node[name=Br, below of=Cl, Halogen] {\NaturalElementTextFormat{35}{79.904}{Br}{Bromine}};
  \node[name=I, below of=Br, Halogen] {\NaturalElementTextFormat{53}{126.9}{I}{Iodine}};
  \node[name=At, below of=I, Halogen] {\NaturalElementTextFormat{85}{210}{At}{Astatine}};
  \node[name=Uus, below of=At, Element] {\SyntheticElementTextFormat{117}{292}{Uus}{Ununseptium}}; 

%% Group 18 - VIIIA
  \node[name=Ne, right of=F, NobleGas] {\NaturalElementTextFormat{10}{20.180}{Ne}{Neon}};
  \node[name=He, above of=Ne, NobleGas] {\NaturalElementTextFormat{2}{4.0025}{He}{Helium}};
  \node[name=Ar, below of=Ne, NobleGas] {\NaturalElementTextFormat{18}{39.948}{Ar}{Argon}};
  \node[name=Kr, below of=Ar, NobleGas] {\NaturalElementTextFormat{36}{83.8}{Kr}{Krypton}};
  \node[name=Xe, below of=Kr, NobleGas] {\NaturalElementTextFormat{54}{131.29}{Xe}{Xenon}};
  \node[name=Rn, below of=Xe, NobleGas] {\NaturalElementTextFormat{86}{222}{Rn}{Radon}};
  \node[name=Uuo, below of=Rn, Nonmetal] {\SyntheticElementTextFormat{118}{294}{Uuo}{Ununoctium}}; 

%% Period
  \node[name=Period1, left of=H, PeriodLabel] {1};
  \node[name=Period2, left of=Li, PeriodLabel] {2};
  \node[name=Period3, left of=Na, PeriodLabel] {3}; 
  \node[name=Period4, left of=K, PeriodLabel] {4}; 
  \node[name=Period5, left of=Rb, PeriodLabel] {5};
  \node[name=Period6, left of=Cs, PeriodLabel] {6};
  \node[name=Period7, left of=Fr, PeriodLabel] {7};

%% Group
  \node[name=Group1, above of=H, GroupLabel] {1 \hfill IA};
  \node[name=Group2, above of=Be, GroupLabel] {2 \hfill IIA};
  \node[name=Group3, above of=Sc, GroupLabel] {3 \hfill IIIA};
  \node[name=Group4, above of=Ti, GroupLabel] {4 \hfill IVB};
  \node[name=Group5, above of=V, GroupLabel] {5 \hfill VB};
  \node[name=Group6, above of=Cr, GroupLabel] {6 \hfill VIB};
  \node[name=Group7, above of=Mn, GroupLabel] {7 \hfill VIIB};
  \node[name=Group8, above of=Fe, GroupLabel] {8 \hfill VIIIB};
  \node[name=Group9, above of=Co, GroupLabel] {9 \hfill VIIIB};
  \node[name=Group10, above of=Ni, GroupLabel] {10 \hfill VIIIB};
  \node[name=Group11, above of=Cu, GroupLabel] {11 \hfill IB};
  \node[name=Group12, above of=Zn, GroupLabel] {12 \hfill IIB};
  \node[name=Group13, above of=B, GroupLabel] {13 \hfill IIIA};
  \node[name=Group14, above of=C, GroupLabel] {14 \hfill IVA};
  \node[name=Group15, above of=N, GroupLabel] {15 \hfill VA};
  \node[name=Group16, above of=O, GroupLabel] {16 \hfill VIA};
  \node[name=Group17, above of=F, GroupLabel] {17 \hfill VIIA};
  \node[name=Group18, above of=He, GroupLabel] {18 \hfill VIIIA};

%% Lanthanide
  \node[name=La, below of=Rf, LanthanideActinide, yshift=-1cm] {\NaturalElementTextFormat{57}{138.91}{La}{Lanthanum}};
  \node[name=Ce, right of=La, LanthanideActinide] {\NaturalElementTextFormat{58}{140.12}{Ce}{Cerium}};
  \node[name=Pr, right of=Ce, LanthanideActinide] {\NaturalElementTextFormat{59}{140.91}{Pr}{Praseodymium}};
  \node[name=Nd, right of=Pr, LanthanideActinide] {\NaturalElementTextFormat{60}{144.24}{Nd}{Neodymium}};
  \node[name=Pm, right of=Nd, LanthanideActinide] {\NaturalElementTextFormat{61}{145}{Pm}{Promethium}};
  \node[name=Sm, right of=Pm, LanthanideActinide] {\NaturalElementTextFormat{62}{150.36}{Sm}{Samarium}};
  \node[name=Eu, right of=Sm, LanthanideActinide] {\NaturalElementTextFormat{63}{151.96}{Eu}{Europium}};
  \node[name=Gd, right of=Eu, LanthanideActinide] {\NaturalElementTextFormat{64}{157.25}{Gd}{Gadolinium}};
  \node[name=Tb, right of=Gd, LanthanideActinide] {\NaturalElementTextFormat{65}{158.93}{Tb}{Terbium}};
  \node[name=Dy, right of=Tb, LanthanideActinide] {\NaturalElementTextFormat{66}{162.50}{Dy}{Dysprosium}};
  \node[name=Ho, right of=Dy, LanthanideActinide] {\NaturalElementTextFormat{67}{164.93}{Ho}{Holmium}};
  \node[name=Er, right of=Ho, LanthanideActinide] {\NaturalElementTextFormat{68}{167.26}{Er}{Erbium}};
  \node[name=Tm, right of=Er, LanthanideActinide] {\NaturalElementTextFormat{69}{168.93}{Tm}{Thulium}};
  \node[name=Yb, right of=Tm, LanthanideActinide] {\NaturalElementTextFormat{70}{173.04}{Yb}{Ytterbium}};
  \node[name=Lu, right of=Yb, LanthanideActinide] {\NaturalElementTextFormat{71}{174.97}{Lu}{Lutetium}};

%% Actinide
  \node[name=Ac, below of=La, LanthanideActinide, yshift=-1cm] {\NaturalElementTextFormat{89}{227}{Ac}{Actinium}};
  \node[name=Th, right of=Ac, LanthanideActinide] {\NaturalElementTextFormat{90}{232.04}{Th}{Thorium}};
  \node[name=Pa, right of=Th, LanthanideActinide] {\NaturalElementTextFormat{91}{231.04}{Pa}{Protactinium}};
  \node[name=U, right of=Pa, LanthanideActinide] {\NaturalElementTextFormat{92}{238.03}{U}{Uranium}};
  \node[name=Np, right of=U, LanthanideActinide] {\SyntheticElementTextFormat{93}{237}{Np}{Neptunium}};
  \node[name=Pu, right of=Np, LanthanideActinide] {\SyntheticElementTextFormat{94}{244}{Pu}{Plutonium}};
  \node[name=Am, right of=Pu, LanthanideActinide] {\SyntheticElementTextFormat{95}{243}{Am}{Americium}};
  \node[name=Cm, right of=Am, LanthanideActinide] {\SyntheticElementTextFormat{96}{247}{Cm}{Curium}};
  \node[name=Bk, right of=Cm, LanthanideActinide] {\SyntheticElementTextFormat{97}{247}{Bk}{Berkelium}};
  \node[name=Cf, right of=Bk, LanthanideActinide] {\SyntheticElementTextFormat{98}{251}{Cf}{Californium}};
  \node[name=Es, right of=Cf, LanthanideActinide] {\SyntheticElementTextFormat{99}{252}{Es}{Einsteinium}};
  \node[name=Fm, right of=Es, LanthanideActinide] {\SyntheticElementTextFormat{100}{257}{Fm}{Fermium}};
  \node[name=Md, right of=Fm, LanthanideActinide] {\SyntheticElementTextFormat{101}{258}{Md}{Mendelevium}};
  \node[name=No, right of=Md, LanthanideActinide] {\SyntheticElementTextFormat{102}{259}{No}{Nobelium}};
  \node[name=Lr, right of=No, LanthanideActinide] {\SyntheticElementTextFormat{103}{262}{Lr}{Lawrencium}};

%% Draw dotted lines connecting Lanthanide breakout to main table
  \draw (LaLu.north west) edge[dotted] (La.north west)
        (LaLu.north east) edge[dotted] (Lu.north east)
        (LaLu.south west) edge[dotted] (La.south west)
        (LaLu.south east) edge[dotted] (Lu.south east);
%% Draw dotted lines connecting Actinide breakout to main table
  \draw (AcLr.north west) edge[dotted] (Ac.north west)
        (AcLr.north east) edge[dotted] (Lr.north east)
        (AcLr.south west) edge[dotted] (Ac.south west)
        (AcLr.south east) edge[dotted] (Lr.south east);

%% Legend
  \draw[black, AlkaliMetalFill] ($(La.north -| Fr.west) + (1em,-0.0em)$)
    rectangle +(1em, 1em) node[right, yshift=-1ex]{Alkali Metal};
  \draw[black, AlkalineEarthMetalFill] ($(La.north -| Fr.west) + (1em,-1.5em)$)
    rectangle +(1em, 1em) node[right, yshift=-1ex]{Alkaline Earth Metal};
  \draw[black, MetalFill] ($(La.north -| Fr.west) + (1em,-3.0em)$)
    rectangle +(1em, 1em) node[right, yshift=-1ex]{Metal};
  \draw[black, MetalloidFill] ($(La.north -| Fr.west) + (1em,-4.5em)$)
    rectangle +(1em, 1em) node[right, yshift=-1ex]{Metalloid};
  \draw[black, NonmetalFill] ($(La.north -| Fr.west) + (1em,-6.0em)$)
    rectangle +(1em, 1em) node[right, yshift=-1ex]{Non-metal};
  \draw[black, HalogenFill] ($(La.north -| Fr.west) + (1em,-7.5em)$)
    rectangle +(1em, 1em) node[right, yshift=-1ex]{Halogen};
  \draw[black, NobleGasFill] ($(La.north -| Fr.west) + (1em,-9.0em)$)
    rectangle +(1em, 1em) node[right, yshift=-1ex]{Noble Gas};
  \draw[black, LanthanideActinideFill] ($(La.north -| Fr.west) + (1em,-10.5em)$)
    rectangle +(1em, 1em) node[right, yshift=-1ex]{Lanthanide/Actinide};

  \node at ($(La.north -| Fr.west) + (5em,-15em)$) [name=elementLegend, Element, fill=white]
    {\NaturalElementTextFormat{Z}{mass}{Symbol}{Name}};
  \node[Element, fill=white, right of=elementLegend, xshift=1em]
    {\SyntheticElementTextFormat{}{}{man-made}{}} ;

%% Diagram Title
  \node at (H.west -| Fe.north) [name=diagramTitle, TitleLabel]
    {(Mendeleev's) Periodic Table of Chemical Elements via Ti\emph{k}Z};

\end{tikzpicture}